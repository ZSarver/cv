\documentclass[11pt]{article}
\usepackage[margin=1in]{geometry}
\usepackage{lastpage}
\usepackage{paralist}

\usepackage{fancyhdr}
\setlength{\headheight}{15.2pt}
\pagestyle{fancy}
\fancypagestyle{plain}{ %
  \fancyhf{} % remove everything
  \renewcommand{\headrulewidth}{0pt} % remove lines as well
  \renewcommand{\footrulewidth}{0pt}
}

\begin{document}
\fancyhf{}
\rhead[Zachary Sarver]{\thepage / \pageref{LastPage}}
%\chead{}{}
\lhead[\thepage / \pageref{LastPage}]{Zachary Sarver}

\lfoot[\today]{Statement of Teaching Philosophy}
%\cfoot[Statement of Teaching Philosophy]{Statement of Teaching Philosophy}
\rfoot[Statement of Teaching Philosophy]{\today}

\title{Statement of Teaching Philosophy}
\author{Zachary Sarver}
\date{}
\maketitle

One of the biggest challenges faced by mathematics educators is motivation. For
better or worse, few people are intrinsically motivated to study math. As a
teacher, my challenge is threefold. I must
\begin{inparaenum}[\itshape a\upshape)]
\item effectively communicate course material;
\item motivate students without oversimplifying course material; and
\item challenge students without discouraging them.
\end{inparaenum}
I have found no collection of methods more effective at tackling all three of
these challenges simultaneously than the methods of inquiry-based learning
(IBL.)

\section{Effectively communicating course material}

As a student of mathematics, I have found that there is no more powerful tool
for learning than \emph{context}, which is exactly what I try to give my
students. By context I mean various facts, useful or at least amusing, about the
course material that can serve as an anchor in the students' minds, something
that holds knowledge in place. Context is a mnemonic device which at the
very least improves student recall, the foundation upon which
understanding rests.

I've used various bits of context to make things stick. Often context is
historical. Newton and Leibniz's dramatic feud frequently serves as a
context in a first calculus course. Context is motivated by applications. To
name one example, modern computer graphics depend heavily on linear algebra,
giving any student who has ever played a video game a reason to
remember. Context can also be a silly anecdote or a bit of trivia about
historical terminology or notation. In my experience, most students do think
that fluxion is a better word than derivative, and mathematical language today
is worse off for it.

Context runs in parallel to course material itself, which I have delivered both
in lecture style, as a first-year TA who didn't know any better, and using
various IBL methods. While a well-written and well-delivered lecture \emph{can}
be captivating, it takes a truly talented speaker to hold the attention of every
teenager in the room while talking about Taylor polynomials. Lecture-style
teaching is also high-variance. A single missed cup of coffee can throw off your
rhythm enough to let students' attention drift.

I find IBL methods much more reliable. Active learning, guided worksheets,
student presentations, students working examples on the chalkboard, and simple
question-heavy Socratic lecturing have all been effective for my classes. More than
anything, I pride myself on being adaptive and flexible in my teaching. I do not
believe that any one IBL method is the One True Method, and I will use whatever
method is most appropriate, depending on course material, student ability, and
student confidence. The most important thing is that students learn best by
doing, not by watching.

\section{Motivating without Oversimplifying}

There is an easy way to motivate students. Make the material easy enough to be
understood immediately. Make all the homework simple, and all the tests easy,
all but guaranteeing good grades. This is a bad way to motivate students.

My primary motivational technique is homework assignments that cover in some
detail relevant applications of the material. For students who are not
motivated by grades or the material itself, this works well. 

Modern students tend to heavily use computers while doing their homework. There
are several websites, which I will not name here, that host solutions to
exercises in undergraduate math textbooks. The typical workflow for a poor
student, as I understand it, is to attempt a homework problem, get frustrated,
look up the solution, and then assume after reading the solution that they have
mastered the skills needed to work this and all similar homework problems. For
this reason, it is vitally important to write your own homework problems.

Even then, online tools such as Wolfram Alpha make problems that require only
straightforward computation or algebra trivial. More tech-savvy students might
have Maple, Mathematica, or SAGE installed on their personal computers. Because
students are going to use these tools anyway, I have decided to embrace it and
design homework assignments with computers in mind. My favorite thing to do is
to ask for several computations, and then prompt the students to generalize the
computations to a relevant theorem.

\section{Challenging without Discouraging}

There is a small mammal, adults weighing two and a half pounds on average, with
long ears, bright eyes, a love for carrots, and a nonsensical association with
Easter eggs. Let us call this animal a student.

It is a sunny spring day with clear blue skies. Warm, but not hot. You are at a
window overlooking your back yard, with the window open to enjoy the
breeze. From the bushes next to the fence, there is a slight rustle, catching
your attention just as a student comes out from under cover. It stands for a bit
on its hind legs, snuffling at the air, surveying its immediate vicinity for
danger before hopping over to delicately nibble at a dandelion. Across the
street, a car door slams, and the student bolts to the safety of its den.

Students are skittish. Failure is discouraging, and hence challenge is
scary. This is unfortunate, because students learn when they are
challenged. There is an art to writing assignments and assessments that are
challenging, but not discouraging. IBL techniques achieve this very well. Active
learning in particular gives students opportunity to practice and learn while
under minimal pressure.

As an undergraduate, in the small handful of classes in which I was not
motivated by the material itself, I was motivated by being graded. I was never
discouraged by apparent challenge, because it was a sink or swim situation. If I
have a choice between a guaranteed failure, or a chance at success, I always
grasp at the chance of success. As a naive instructor, I assumed that the
material or grades would also be the only thing my students care about, and I
was wrong.

Despite my intuitions to the contrary, I have found that being likable is one of
those auxiliary things that is important to class performance. It's easy for a
student to dismiss an instructor that he or she thinks is a jerk. Thus I work
hard to keep my classroom mannerisms light and friendly, without actually
crossing the line to appear to be the students' friend. This encourages the
students to take on the challenges that I deliver, or at least helps to keep
them from being completely discouraged.

The pressure of assessment is by and large inescapable, which is why I always
make sure to leave ample time for test review to help students feel more
prepared. It is on test review days that I find bringing students to the
chalkboard to work to be most useful. They are not (supposed to be) just
learning the material, so they can approach the chalkboard with some (sometimes
feigned) confidence. Their classmates are also busy working problems, so they
are able to approach the board without fear of being judged by their
peers.

\section{Other Remarks}

I believe it is of vital importance for students to learn form their
mistakes. At the same time that I write an assessment, I also typeset solutions
for that assessment. When it is over and graded and handed back, I post the
solutions online. This way students can learn from their mistakes on problems
that they have already worked to completion.

In the classroom itself, I am currently designing a classroom participation app
that will, in a sense, allow students to present anonymously, removing one of
the barriers to the ``buy-in'' problems that frequently plague IBL teaching
methods.

\end{document}

%%% Local Variables:
%%% mode: latex
%%% TeX-master: t
%%% End:
