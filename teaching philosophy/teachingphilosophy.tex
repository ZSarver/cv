\documentclass[11pt]{article}
\usepackage[margin=1in]{geometry}
\usepackage{lastpage}
\usepackage{paralist}

\usepackage{fancyhdr}
\setlength{\headheight}{15.2pt}
\pagestyle{fancy}
\fancypagestyle{plain}{ %
  \fancyhf{} % remove everything
  \renewcommand{\headrulewidth}{0pt} % remove lines as well
  \renewcommand{\footrulewidth}{0pt}
}

\begin{document}
\fancyhf{}
\rhead[Zachary Sarver]{\thepage / \pageref{LastPage}}
%\chead{}{}
\lhead[\thepage / \pageref{LastPage}]{Zachary Sarver}

\lfoot[\today]{Statement of Teaching Philosophy}
%\cfoot[Statement of Teaching Philosophy]{Statement of Teaching Philosophy}
\rfoot[Statement of Teaching Philosophy]{\today}

\title{Teaching Statement}
\author{Zachary Sarver}
\date{}
\maketitle

As a teacher, my challenge is threefold. I must
\begin{inparaenum}[\itshape a\upshape)]
\item effectively communicate course material;
\item motivate students; and
\item challenge students without discouraging them.
\end{inparaenum}
I have found no collection of methods more effective at tackling all three of
these challenges simultaneously than the methods of inquiry-based learning
(IBL.)

As a student of mathematics, I have found that there is no more powerful tool
for learning than \emph{context}, which is exactly what I try to give my
students. By context I mean various facts, useful or at least amusing, about the
course material that can serve as an anchor in the students' minds, something
that holds knowledge in place. Context is a mnemonic device which at the very
least improves student recall, the foundation upon which understanding rests.

There is an easy way to motivate students. Make the material easy enough to be
understood immediately. Make all the homework simple, and all the tests easy,
all but guaranteeing good grades. I prefer to instead give homework
assignments that cover in some detail relevant applications of the
material. Concrete uses of mathematics are a context that can help cement
students' understanding of the material.

Modern students tend to heavily use computers while doing their homework. Online
tools such as Wolfram Alpha make problems that require only straightforward
computation or algebra trivial. More tech-savvy students might have Maple,
Mathematica, or SAGE installed on their personal computers. Because students are
going to use these tools anyway, I have decided to embrace it and design
homework assignments with computers in mind. I have found it particularly
effective to ask for several computations, and then prompt the students to
generalize the computations to a relevant theorem.

Failure is discouraging, and hence challenge is scary. This is unfortunate,
because students learn when they are challenged, and it is of vital importance
for students to learn from their mistakes. There is an art to writing
assignments and assessments that are challenging, but not discouraging. IBL
techniques achieve this very well, by giving students opportunity to practice
and learn while under fairly minimal pressure.

Active learning, guided work, student presentations, students working
examples on the chalkboard, and simple question-heavy Socratic lecturing have
all been effective for my classes. More than anything, I pride myself on being
adaptive and flexible in my teaching. I do not believe that any IBL method is
the One True Method, and I will use whatever method is most appropriate,
depending on course material, student ability, and student confidence. The most
important thing is that students learn best by doing, not by watching.

I have an interest in not just teaching using the methods I already know, but
also in furthering education. To that end I am currently designing a classroom
participation app that will, in a sense, allow students to present
anonymously. It is my hope that this will remove the ``buy-in'' problem which
can be a barrier to entry in IBL.

\end{document}

%%% Local Variables:
%%% mode: latex
%%% TeX-master: t
%%% End:
