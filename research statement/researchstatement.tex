\documentclass[11pt]{article}
\usepackage[margin=1in]{geometry}
\usepackage{lastpage}
\usepackage{paralist}
\usepackage{amsmath}
\usepackage{amssymb}

\usepackage{fancyhdr}
\setlength{\headheight}{15.2pt}
\pagestyle{fancy}
\fancypagestyle{plain}{ %
  \fancyhf{} % remove everything
  \renewcommand{\headrulewidth}{0pt} % remove lines as well
  \renewcommand{\footrulewidth}{0pt}
}

\begin{document}
\fancyhf{}
\rhead[Zachary Sarver]{\thepage / \pageref{LastPage}}
%\chead{}{}
\lhead[\thepage / \pageref{LastPage}]{Zachary Sarver}

\lfoot[\today]{Research Statement}
%\cfoot[Statement of Teaching Philosophy]{Statement of Teaching Philosophy}
\rfoot[Research Statement]{\today}

\title{Research Statement}
\author{Zachary Sarver}
\date{}
\maketitle

According to my mother, my answer to, ``what do you want to be when you grow
up?'' was, from the time I was old enough to comprehend the question, ``a
scientist.'' Before that it was, ``a lizard.'' In a very real sense I have
always been interested in discovery. 

As an undergraduate, I started out as a computer science major, only picking up
mathematics as a second major in my sophomore year. I was fascinated with data
structures, particularly trees and graphs, which can have remarkable properties
when used wisely. As I progressed in my mathematics classes, I discovered that
mathematical structures are also fascinating, and eventually developed an
interest in algebraic structures in particular.

\section{Undergraduate research}

In my undergraduate studies I primarily concentrated on computer science
research. I worked as an undergraduate research assistant at Jacksonville State
University in fast encoding of multi-reference frame video encoding under
Dr. Monica Trifas and Dr. Ming Yang. Briefly, modern video streaming is done by
only occasionally transmitting a full video frame, called a reference frame. The
rest of the frames are divided into small square blocks and estimated motion
vectors from these small blocks to corresponding blocks in the nearest reference
frame are transmitted. A multi-reference frame video encoder uses multiple
reference frames to estimate these motion vectors, which results in
higher-quality video, but encoding that is potentially orders of magnitude
slower.

In the last summer at Jacksonville State University, I traveled to Utah State
University to work on a summer research project with Dr. Xiaojun Qi in computer
vision, specifically image classification. We used a support vector machine to
train the computer to classify images based on content, e.g. contains flags
vs. contains birds.

\section{Lie theory}

My dissertation work and current research are in the area of Lie theory,
particularly in semisimple Lie groups.

A \emph{Lie group} is a group that also has the structure of a smooth manifold
such that both multiplication and inversion of group elements are smooth maps. A
smooth manifold itself is a topological space that is locally homeomorphic to a
Euclidean space \( \mathbb{R}^n \),
and for any two local homeomorphisms \( \varphi_1 \)
and \( \varphi_2 \)
whose domains intersect,
\( \varphi_1 \circ \varphi_2^{-1}: \mathbb{R}^n \to \mathbb{R}^n \)
must have partial derivatives of all orders. In my research I have restricted my
attention to finite-dimensional Lie groups.

To every Lie group $G$ there is an associated \emph{Lie algebra} \(
\mathfrak{g} \), isomorphic as a vector space to the tangent space at the identity
of $G$, and having an additional nonassociative algebra structure given by the
Lie bracket. Every Lie algebra is isomorphic to a matrix algebra, whwere \(
[X,Y] = XY - YX \) where juxtaposition is ordinary matrix multiplication. 

A Lie algebra is \emph{simple} if it is nonabelian and its only ideals are
itself and the trivial Lie algebra. A Lie algebra is \emph{semisimple} if it is
the direct sum of simple Lie algebras. A Lie group is semisimple if its
associated Lie algebra is semisimple.

A semisimple Lie group admits a preorder, called Kostant's preorder, and
inequalities with Kostant's preorder are primarily where my interest
lies. Kostant's preorder has a very lengthy and technical definition involving
the Iwasawa decomposition, complete multiplicative Jordan decomposition, and
vector log-majorization, and I will omit the definition here. It is, however,
worth note that many inequalities involving matrices and vector majorization
have been generalized to semisimple Lie groups, and many more seem to be good
candidates for doing so.

\section{Matrix theory}

Here's some specific stuff about Matrix theory, specifically future work in
matrix means and generalizing audenaert's result

\section{Topological groups}

Here's some specific stuff about topological groups, and my interest in
integration.

\section{Multilinear algebra}

Here's some specific stuff about multilinear algebra, and my interest in
symmetry classes and tensor rank, as well as obstacles to studying tensor
rank. (I.e. algebraic geometry)

\section{Scholarship of education}

IBL stuff probably.

\section{Other interests}

\begin{enumerate}
\item Category theory
\item Computability
\item Quantum computation
\item Programming languages
\end{enumerate}

\end{document}
%%% Local Variables:
%%% mode: latex
%%% TeX-master: t
%%% End:
