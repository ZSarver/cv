\documentclass[11pt]{article}
\usepackage[margin=1in]{geometry}
\usepackage{lastpage}
\usepackage{paralist}
\usepackage{amsmath}
\usepackage{amssymb}
\usepackage[style=alphabetic]{biblatex}
\usepackage{hyperref}

\usepackage{fancyhdr}
\setlength{\headheight}{15.2pt}
\pagestyle{fancy}
\fancypagestyle{plain}{ %
  \fancyhf{} % remove everything
  \renewcommand{\headrulewidth}{0pt} % remove lines as well
  \renewcommand{\footrulewidth}{0pt}
}

\addbibresource{refs.bib}

\begin{document}
\fancyhf{}
\rhead[Zachary Sarver]{\thepage / \pageref{LastPage}}
%\chead{}{}
\lhead[\thepage / \pageref{LastPage}]{Zachary Sarver}

\lfoot[\today]{Research Statement}
%\cfoot[Statement of Teaching Philosophy]{Statement of Teaching Philosophy}
\rfoot[Research Statement]{\today}

\title{Research Statement}
\author{Zachary Sarver}
\date{}
\maketitle

According to my mother, my answer to, ``what do you want to be when you grow
up?'' was, from the time I was old enough to comprehend the question, ``a
scientist.'' I watched \textit{Bill Nye the Science Guy} on TV every single
chance I got, and devoured every single science book I could get my hands on. In
a very real sense I have always been interested in discovery, although my
interests turned to computer science and mathematics only as a young adult.

As an undergraduate, I started out as a computer science major, only picking up
mathematics as a second major in my sophomore year. I was fascinated with data
structures, particularly trees and graphs, which can have remarkable properties
when used wisely. As I progressed in my mathematics classes, I discovered that
mathematical structures are also fascinating, and eventually developed an
interest in algebraic structures in particular. My published results and current
research are specifically in Lie theory.

\section{Undergraduate research}

In my undergraduate studies I primarily concentrated on computer science
research. I worked as an undergraduate research assistant at Jacksonville State
University in fast encoding of multi-reference frame video encoding under
Dr. Monica Trifas and Dr. Ming Yang. Briefly, modern video streaming is done by
only occasionally transmitting a full video frame, called a reference frame. The
rest of the frames are divided into small square blocks and estimated motion
vectors from these small blocks to corresponding blocks in the nearest reference
frame are transmitted. A multi-reference frame video encoder uses multiple
reference frames to estimate these motion vectors, which results in
higher-quality video, but encoding that is potentially orders of magnitude
slower.

In my last summer at Jacksonville State University, I traveled to Utah State
University to work on a summer research project with Dr. Xiaojun Qi in computer
vision, specifically image classification. We used a support vector machine to
train the computer to classify images based on content, e.g. contains flags
vs. contains birds.

\section{Lie theory}

My dissertation work, current research, and published results are in the area of
Lie theory, particularly in semisimple Lie groups.

A \emph{Lie group} is a group that also has the structure of a smooth manifold
such that both multiplication and inversion of group elements are smooth maps. A
smooth manifold itself is a topological space that is locally homeomorphic to a
Euclidean space \( \mathbb{R}^n \),
and for any two local homeomorphisms \( \varphi_1 \)
and \( \varphi_2 \)
whose domains intersect,
\( \varphi_1 \circ \varphi_2^{-1}: \mathbb{R}^n \to \mathbb{R}^n \)
must have partial derivatives of all orders. In my research I have restricted my
attention to finite-dimensional Lie groups.

To every Lie group $G$ there is an associated \emph{Lie algebra} \(
\mathfrak{g} \), isomorphic as a vector space to the tangent space at the identity
of $G$, and having an additional nonassociative algebra structure given by the
Lie bracket. Every Lie algebra is isomorphic to a matrix algebra such that \(
[X,Y] = XY - YX \) where juxtaposition is ordinary matrix multiplication. 

A Lie algebra is \emph{simple} if it is nonabelian and its only ideals are
itself and the trivial Lie algebra. A Lie algebra is \emph{semisimple} if it is
the direct sum of simple Lie algebras. A Lie group is semisimple if its
associated Lie algebra is semisimple.

A semisimple Lie group admits a preorder, called Kostant's preorder, and
inequalities with Kostant's preorder are primarily where my interest
lies. Kostant's preorder has a very lengthy and technical definition involving
the Iwasawa decomposition, complete multiplicative Jordan decomposition, and
vector log-majorization, and so I will omit it here. It is, however,
worth noting that many inequalities involving matrices and vector majorization
have been generalized to semisimple Lie groups, and many more seem to be good
candidates for doing so.

Inverse limits of Lie groups have been studied, and only recently has a
categorial description of generalized inverse limits been given. I am currently
very interested in studying the potential of generalized inverse limits of Lie
groups.

\section{Matrix theory}

A matrix can be more than a convenient representation of a linear map, as
various spaces of matrices have rich structure as groups, algebras, topologies,
and manifolds.

A complex \( n \times n \)
matrix is called \emph{positive definite} if it is hermitian and all its
eigenvalues are real and greater than $0$. The set \( \mathbb{P}_n \)
of \( n \times n \)
positive definite matrices does not form a group, as the product of two positive
definite matrices may not be hermitian. \( \mathbb{P}_n \)
does, however, admit the structure of a Riemannian manifold, and for
\( A,B \in \mathbb{P}_n \),
and the geometric mean
\( A \sharp B = A^{1/2}(A^{-1/2}BA^{-1/2})^{1/2}A^{1/2} \)
is in fact the midpoint of the geodesic connecting $A$ and $B$. Note that if \(
n = 1 \), i.e. $A$ and $B$ are positive numbers, then \( A \sharp B \) is the
ordinary geometric mean.

Audenaert recently proved in \cite{au15} that for $k$-many commuting pairs of
positive definite matrices \( A_i \) and \( B_i \) and for any unitarily
invariant norm \( \|\cdot\| \), \[ \left\| \sum_{i=1}^kA_iB_i\right\| \leq
  \left\|\left(\sum_{i=1}^kA_i^{1/2}B_i^{1/2}\right)^2\right\| \leq
  \left\|\left( \sum_{i=1}^kA_i \right)\left( \sum_{i=1}^kB_i\right)\right\| \]

It has been conjectured that a similar inequality holds in the noncommuting case
when matrix multiplication is replaced with the geometric mean.

\section{Topological groups}

Much like a Lie group is a group that is a differentiable manifold, a
topological group is a group with a simpler structure: it must also be a
topological space with multiplication and inversion continuous. A topological
space, and hence topological group, is locally compact if every point has a
compact neighborhood.

In \cite{ea96}, it is proved that every locally compact topological group
has something resembling a differential structure, in that a derivative may be
defined, four of them, in fact, that behave remarkably similar to the ordinary
derivative of calculus. 

I lead a two-semester seminar, along with my friend and colleague Alan Bertl, to
investigate the subject of integration in topological groups, during which we
worked through a large swath of Pontryagin's classic text \cite{po86}.

Much about integration in topological groups remains intriguing and mysterious
to me, so I am keenly interested in pushing the boundaries of this field.

\section{Multilinear algebra}

A multilinear map \( \alpha: V_1 \times \dots \times V_m \to W \) is a map that
is linear in each coordinate. The difficulty in studying multilinear maps is
that they are not linear maps, but they are easier to work with when viewed
through the lens of tensors. There are several constructions of the tensor
product that are all isomorphic for vector spaces over \( \mathbb{R} \) or \(
\mathbb{C} \), so I will give a more concise one here. Let \( T: V_1 \dots
\times \dots V_m \to W \) be a multilinear map satisfying \(
\text{span}(\text{Im } T) = \prod_{i=1}^m\text{dim } V_i \). Then \(
\bigotimes_{i=1}^m V_i = \text{span}(\text{Im } T) \). An element of the form \(
v_1 \otimes \dots \otimes v_m \) with \( v_i \in V_i \) for all $i$ is called a
\emph{pure tensor}, and a general element of \( \bigotimes_{i=1}^m V_i \) is a
linear combination of pure tensors, hence we may conveniently define things on
pure tensors and extend linearly.

Let \( S_m \)
be the symmetric group on $m$ symbols. Let $G$ be a subgroup of \( S_m \),
and let $G$ act on pure tensors by
\( P(\sigma)(v_1 \otimes \dots \otimes v_m) = v_{\sigma^{-1}(1)} \otimes \dots
\otimes v_{\sigma^{-1}(m)} \),
where \( \sigma \)
acts on \( 1, \dots, m \)
by permutation. Extend this action linearly to get a group action, which we
denote by the operator \( P(\sigma) \), on the entire tensor space.

Let \( \chi \) be an irreducible character of $G$. The operator \( T(G;\chi) =
\frac{\chi(e)}{|G|}\sum_{\sigma\in G}\chi(\sigma)P(\sigma) \) is called the
\emph{symmetrizer with respect to $G$ and \( \chi \)}. Symmetrizers are
projections, and the image of a symmetrizer is called a \emph{symmetry class of
tensors}.

Symmetry classes of tensors are of varied combinatorial and algebraic
interest. I have written routines for computations involving symmetry classes of
tensors in the programming language GAP. Although there are many open problems
in multilinear algebra, I am currently most interested in porting these routines
to a speedier general purpose programming library to make them more readily
available.

Tensors have seen a wide range of applications in the last decade or so, most
notably in bioinformatics, computer vision, and numerical linear algebra. Of
particular interest is the rank $r$ of a tensor, the lowest $r$ such that a
tensor may be expressed as \( \sum_{i=1}^r\alpha_i v_1^i \otimes \dots \otimes
v_m^i \). There are a number of different kinds of rank that are also of
interest, and currently tensor rank is best studied with the methods of
algebraic geometry.

\section{Other interests}

Although I have not done any research in the following areas, I am also
interested in:

\begin{itemize}
\item Category theory
\item Computability
\item Quantum computation
\item Programming language design
\end{itemize}

\printbibliography

\end{document}
%%% Local Variables:
%%% mode: latex
%%% TeX-master: t
%%% End:
