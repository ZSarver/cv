\documentclass{article}
\usepackage[margin=1in]{geometry}
\usepackage{multicol}
\usepackage{float}
\usepackage{fancyhdr}
\usepackage{enumitem}
\usepackage{amsmath}
\usepackage{amssymb,amscd}

\usepackage[pdfpagelabels]{hyperref}
\hypersetup{colorlinks=true,linkcolor=black}

\pagestyle{fancy} %\headheight 14.49998pt

\rhead{Sarver \thepage}
\chead{}
\lhead{Curriculum Vitae}
\rfoot{\today}
\cfoot{}
\lfoot{}    

\newcommand{\HRule}{\rule{\linewidth}{0.5mm}}

\begin{document}

\thispagestyle{empty}

\begin{center}
{\Large Zachary Gerald Sarver} \\[.5pc]
Auburn University Department of Mathematics \& Statistics \\
\href{mailto:zack@zacharysarver.net}{zack@zacharysarver.net} $;$
\href{mailto:Zachary.Sarver@gmail.com}{Zachary.Sarver@gmail.com} $;$
\href{mailto:zgs0004@tigermail.auburn.edu}{zgs0004@tigermail.auburn.edu} \\
Websites: \url{http://zacharysarver.net/} \& \url{http://zacksarver.rocks/}
\end{center}

\noindent {\large \bf Education} \\*[-.8pc]
\rule{\textwidth}{.1pt} \\
{\bf PhD, Auburn University}\hfill {\it 2016} \\
Dissertation: \textit{Extension of Monotonicity Results to Semisimple Lie Groups} \\
\\
{\bf Bachelor of Science in Mathematics and Computer Science,\\ Jacksonville State
University}\hfill {\it 2009}\\
Distinguished Graduate in Mathematics\\
Special Honors in Computer Science\\
Magna Cum Laude\\

\

\noindent {\large \bf Research Interests} \\*[-.8pc]
\rule{\textwidth}{.1pt} 
\begin{multicols}{2}
\begin{itemize}[noitemsep]
\item Lie Theory
\item Matrix Theory
\item Category Theory
\item Linear and Multilinear Algebra
\item Computability
\item Functional Reactive Programming
\end{itemize}
\end{multicols}

\noindent {\large \bf Publications} \\*[-.8pc]
\rule{\textwidth}{.1pt} \\
\begin{itemize}[noitemsep]
\item \emph{An Extension of Wang-Gong Monotonicity Result to Semisimple Lie
    Groups}. Accepted for publication in \textit{Special Matrices}
\item Currently untitled paper on Lie groups. Writing
\end{itemize}
\vspace{1ex}
\noindent {\large \bf Talks and Presentations} \\*[-.8pc]
\rule{\textwidth}{.1pt} \\
\textbf{The Action of \( \mathfrak{sl}_2(\mathbb{C}) \) and Independent Sets}
\hfill \textit{April 05, 2014}\\
This talk was given at the 2014 Graduate Student Combinatorics Conference. It
detailed a particular action of the Lie algebra
\( \mathfrak{sl}_2(\mathbb{C}) \) on graphs and contained results related to
this action.
\\ \\
\textbf{The Very Basics of Multilinear Algebra} \hfill \textit{October 08,
  2014}\\
This talk was on definitions and basic theorems in multilinear maps between
complex vector spaces, tensor spaces, matrix representation of tensors and
tensor representation of matrices, and tensor products of $R$-modules. It was
aimed at first year graduate students as an introduction to the topic of
multilinear algebra.
\\ \\
\textbf{The Category \textup{Hask}} \hfill \textit{November 11, 2014}\\
This talk described the category \textbf{IdealHask} whose objects are Haskell
types without bottom values and morphisms are Haskell functions. It described
some category theoretic constructions in \textbf{IdealHask} as well as
endofunctors and natural transformations between endofunctors. It remarked on
the differences between \textbf{IdealHask} and \textbf{Hask}, the latter's types
containing bottom values.
\\ \\
\textbf{On Kostant's Preorder} \hfill \textit{April 13, 2015}\\
This talk detailed the background material needed to understand the definition
of Kostant's preorder on semisimple Lie groups: the Cartan decomposition, the
Iwasawa decomposition, the complete multiplicative Jordan decomposition, and the
Weyl group.
\\ \\
\textbf{Lie Groups and Why You Might Care About Them} \hfill \textit{September
  8, 2015} \\
This talk was given in the Auburn COSAM Interdisciplinary Colloquium. It was a
short talk on the basic definitions of Lie theory given to a general audience of
graduate students and undergraduate students in the College of Science and
Mathematics (COSAM.) This colloquium is intended to foster interested in
cross-discipline collaboration of graduate students.\\
\\ \\
\textbf{Tensor Products of Graphs: A Categorial Perspective} \hfill
\textit{December 15th, 2015} \\
This talk was given at the 2nd annual Southern Combinatorics, Graph Theory, and
Game Theory Mini-Conference held at Lamar University in Beaumont, TX. The aim
was to motivate the study of the tensor products of graphs from an algebraic and
category-theoretic perspective. This talk introduced the concept of a category,
the category of undirected graphs, the binary categorial product, and the tensor
product of graphs. It was shown that the binary categorial product in the
category of undirected graphs is the
tensor product of graphs. \\
\\ \\
\textbf{Extension of Wang-Gong monotonicity results in semisimple Lie groups}
\hfill \textit{December 19th, 2015} \\
This talk was given at the 5th International Conference on Matrix Analysis and
Applications held at Nova Southeastern University in Ft. Lauderdale, FL. It gave
extensions of both eigenvalue and singular value type inequalities to semisimple
Lie groups with Kostant's preorder. \\
\\
\noindent {\large \bf Teaching Experience, Auburn University} \\*[-.8pc]
\rule{\textwidth}{.1pt} \\
% \textbf{About Auburn University} \\ Auburn University is a public
% university with approximately $25,000$ students.  Auburn University is
% a land grant university that has a high undergraduate population,
% which is renowned for its engineering program, most notably its
% Aerospace Engineering Program.  In addition to research and
% instruction, Auburn University prides itself on its outreach programs.\\
% \\
{\bf Graduate Teaching Assistant of Mathematics} \hfill {\it Spring 2010 to present\/}
\begin{itemize}[noitemsep]
\item Oversaw computer-based pre-calculus classes.
\item Aided with lecturing and grading in a sophomore
  linear algebra class.  
\item Worked as a tutor for Pre-Calculus, Calculus I, Calculus II, and
  Calculus III.  
\item Served as the instructor of record for six courses:
\begin{table}[H]
\begin{center}
\begin{tabular}{lll}
Math $1130$ & Pre-Calculus & Summer $2013$ \\
Math $1680$ & Calculus with Business Applications I & Spring $2012$ \\
Math $1610$ & Calculus I & Fall $2011$, Fall $2012$, Spring $2015$ \\
Math $1620$ & Calculus II & Spring $2013$, Fall $2013$, Fall $2015$ \\
Math $2630$ & Calculus III & Spring $2014$, Summer $2014$ \\
Math $2660$ & Linear Algebra & Summer $2015$ \\
[-1pc]
\end{tabular}
\end{center}
\caption{Courses Taught at Auburn University}%ascending by year mostly
\end{table}
\item Wrote tests, assignments, and finals for almost every course taught.
\item Worked with several students so that they could receive honors credit for
  an otherwise non-honors class.
\item Had excellent student evaluations. Detailed records available upon
  request.
\end{itemize}

\noindent{\bf Calculus Tutor} \hfill {\it Fall 2010\/} \\
Staffed weekly university ``Help Room'' where students could go for help in
their undergraduate Calculus courses. \\ 

\noindent {\bf Pre-Calculus} \hfill {\it Spring 2010 and Summer 2013} \\
Oversaw two sections of the computer based Pre-Calculus classes at Auburn
University, holding weekly help sessions as well as additional review sessions
for the midterm and final.  Answered questions and taught class as the
students completed the assignments on the computer. \\
Served as the instructor of record in 2013, during which I covered polynomials
and factoring, rational functions, conic sections, and complex numbers.
\\

\noindent\textbf{Calculus with Business Applications I} \hfill \textit{Spring 2012}\\
Covered limits, derivatives of algebraic, exponential, and logarithmic
functions, anti-derivatives, the definite integral, multivariate functions,
partial derivatives, the method of Lagrange multipliers, and applications of the
preceding to business and economics.
\\

\noindent{\bf Calculus I} \hfill {\it Fall 2011, Fall 2012, Spring 2015}\\
Covered limits; the derivatives of algebraic, trigonometric, exponential, and
logarithmic functions; applications of the derivative; anti-derivatives; the
definite integral and its applications to area problems; and the Fundamental
Theorem of Calculus. \\ For Fall 2012 I taught the class in an active learning
style. The philosophy of active learning is that students should learn by
doing. I prepared worksheets for teaching the entire course via guiding
questions and the Socratic method.

\

\noindent {\bf Calculus II} \hfill {\it Spring 2013, Fall 2013, and Fall 2015}\\
Covered techniques of integration, applications of the
integral, vectors, lines and planes in space, and infinite sequences
and series.  

\

\noindent {\bf Calculus III} \hfill {\it Spring 2014 and Summer 2014}\\
Covered multivariable calculus, including vector-valued
functions, partial derivatives, multiple integration, and vector
calculus. 

\

\noindent {\bf Linear Algebra} \hfill {\it Fall 2010 and Summer 2015}\\
Taught the majority of lectures for one section of a sophomore linear algebra
class as well as graded for said class.\\ Served as the instructor of record in
2015, during which I covered systems of linear equations, matrix algebra, vector
spaces, basis and dimension, change of basis, and eigenvalues.

\

\noindent {\bf Math Summer Bridge Program} \hfill {\it Summer 2012}\\
The COSAM Summer Bridge Program was an intensive four-week residential
program for talented and highly motivated minority students (with
respect to the sciences) who wanted to get a head start in their college career.\\
\\
\noindent {\large \bf Work and Research at Jacksonville State University} \\*[-.8pc]
\rule{\textwidth}{.1pt} \\
{\bf Tutor at ACE Tutoring Center} \hfill \textit{Fall 2006 -- Fall 2008}\\
The Academic Center for Excellence is a Jacksonville State University program
for providing free tutoring to JSU undergrades. Tutored undergraduate
mathematics and computer science students primarily, as
well as the occasional English Composition student.\\
\\
{\bf Undergraduate Research Assistant} \hfill {\it Fall 2007 -- Fall 2008} \\
Aided Dr. Monica Trifas and Dr. Ming Yang in their research in multi-reference
frame video encoding. Reviewed code and implemented functions in C.\\

\noindent{\large \bf Service and Extracurriculars} \\*[-.8pc]
\rule{\textwidth}{.1pt} \\
\textbf{AMP'd Challenge} \hfill \textit{Fall 2015}\\
Helped to design puzzles, worked with students during the event, and played a
character during the event. AMP'd Challenge is an outreach program for middle
school students, giving them
mathematical challenges in the guise of puzzles.  \\
\\
\textbf{Web Developer at Mathematical Puzzle Programs} \hfill \textit{Fall 2015}\\
Maintaining and updating the official MaPP website at
\url{http://mappmath.org}. \textbf{Ma}thematical \textbf{P}uzzle \textbf{P}rograms
is an extension of AMP'd Challenge to bring mathematical challenges, cleverly
disguised as puzzles, to high- and middle-school students nationwide.\\
\\
\textbf{Auburn Puzzle Parties}\\
Auburn has a thriving puzzle and puzzlehunt community of people who like to
design and solve puzzles. In addition to participating in a number of puzzle
events, I have also designed puzzles, organized events, playtested puzzles, and
played villainous characters.
\begin{itemize}[noitemsep]
\item \textbf{Auburn Puzzle Party 4: Puzzle Patrol} Playtester and
  organizational assistant. \hfill \textit{Fall 2010}
\item \textbf{Auburn Puzzle Party 5: Puzzle Patrol II} Puzzle designer,
  playtester, villain, and organizational assistant. \hfill \textit{Fall 2012}
\item \textbf{Puzzle Potluck 4} Organizer, puzzle designer, and
  participant. \hfill \textit{Summer 2013}
\end{itemize}

\

\noindent {\large \bf Additional Skills and Work Experience} \\*[-.8pc]
\rule{\textwidth}{.1pt} \\
{\bf Computer Languages and C Libraries} \\
Proficient in Haskell, C, C++, and Python. Some experience with Java,
Objective-C, Go, Ruby, GAP, Zilog Z80 Assembly, and MOS 6502 Assembly. Familiar
with the SDL multimedia library, the LAPACK numerical linear algebra library,
and various POSIX libraries including pthreads. \\
\newpage
\noindent \textbf{Programmer at Envizions} \hfill \textit{2008-2009}\\
Worked on a custom user interface and associated daemons and helper
programs in C, C++, and Ruby. Envizions was a startup based in Anniston, AL that
produced Linux-based set-top boxes for enjoying games and media on your TV.

\

\noindent {\large \bf Awards, Honors, and Special Programs} \\*[-.8pc]
\rule{\textwidth}{.1pt} \\
{\bf Winner of the City of Auburn Municipal Hackathon} \hfill {\it Fall 2014}\\
For \href{https://github.com/IsToomersCornerBeingRolledRightNow}{``Is Toomer's Corner Being Rolled?''}, a webapp that uses image recognition
techniques on still images periodically captured from a municipal webcam to determine
if Toomer's Corner is being rolled, an Auburn football victory tradition.
\\ \\


\end{document}
%%% Local Variables:
%%% mode: latex
%%% TeX-master: t
%%% End:
